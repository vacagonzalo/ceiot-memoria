\chapter{Introducción específica} % Main chapter title

\label{Chapter2}

%----------------------------------------------------------------------------------------
%	SECTION 1
%----------------------------------------------------------------------------------------
Este capítulo trata sobre los recursos tecnológicos de terceros que fueron utilizados en la producción del trabajo.

% Descripción de las plataformas tecnológicas que sostienen al trabajo
\section{Tecnologías utilizadas}

% docker
\subsection{Orquestación}
Para crear una arquitectura de microservicios se utilizó Docker, que es un software que permite el uso y creación de contenedores de Linux.
Un contenedor es una unidad que empaqueta el código de un programa junto con sus dependencias para aislar su funcionamiento.
Para crear un contenedor, el motor de Docker se vale de el concepto de imagen.
Las imágenes son entidades inmutables que cumplen la función de ser plantillas para crear contenedores.
Se puede visualizar a las imágenes como capturas del estado de un contenedor y a partir de estas capturas se pueden instanciar nuevos contenedores.

Los contenedores cumplen la función de ser abstracciones de la capa de aplicación de los sistemas Linux, como se puede visualizar en la figura \ref{fig:ch2WhatContainer}.
Varios contenedores pueden correr en el mismo ordenador como procesos aislados en el espacio de usuario sin generar ningún tipo de interferencias entre sí.

La principal diferencia con las máquinas virtuales, es que estas son una abstracción del hardware del ordenador, transformando una única computadora en varios servidores.
Los contenedores, en cambio, utilizan el kernel del sistema operativo del ordenador físico.
Es decir, no se abstrae un kernel, solo el espacio de aplicación o de usuario.

% docker files
Docker puede ser utilizado para construir imágenes definidas por el usuario.
Para lograrlo se usa un Dokerfile, que es un documento de texto que contiene todos los comandos que un usuario utilizaría para ensamblar una imagen.
El programador puede correr automáticamente una serie de comandos en sucesión al ejecutar una única orden sobre el Dockerfile.

Otra capacidad adicional es subir la imagen a un repositorio para ser descargada directamente sobre el entorno de producción.
De esta manera se facilita el despliegue de la aplicación, por lo que sólo resta orquestar los contenedores para que trabajen en conjunto. 

% docker compose
La orquestación necesaria para la etapa de despliegue se logra utilizando Docker Compose.
Según se indica en su documentación \citep{WEBSITE:WhatDockerCompose}, es una herramienta para definir y correr aplicaciones de Docker de múltiples contenedores.
Permite utilizar un archivo \emph{YAML} para configurar los servicios de la aplicación.
Con esta tecnología se pueden crear y comenzar todos los servicios de la configuración utilizando un único comando y tener orquestada la solución.

\begin{figure}[h]
	\centering
	\includegraphics[width=\textwidth]{./Figures/ch2DockerContainer.png}
	\caption{Arquitectura de Docker. \citep{WEBSITE:WhatContainer}}
	\label{fig:ch2WhatContainer}
\end{figure}

\subsection{Tecnologías de red}
% nodejs
Este trabajo utiliza tecnologías web y la plataforma utilizada para implementarlas fue Nodejs, que es un servidor asincrónico y orientado a eventos que ejecuta JavaScript.
Su arquitectura se puede visualizar en la figura \ref{fig:ch2Nodejs}.
La orientación a eventos se consideró como una ventaja frente a las aplicaciones de concurrencia de múltiples hilos en el sistema operativo, principalmente porque no se utilizan candados y no existe la posibilidad de bloquear el servidor.
Una particularidad adicional es que Nodejs fue diseñado para construir aplicaciones de red escalables y provee un gestor de paquetes muy fácil de utilizar.
Por estas razones y porque además fue una plataforma estudiada en la especialización, es que se la eligió para formar parte del trabajo.

\begin{figure}[h]
	\centering
	\includegraphics[width=\textwidth]{./Figures/ch2Nodejs.jpeg}
	\caption{Arquitectura de Nodejs. \citep{WEBSITE:nodejs}}
	\label{fig:ch2Nodejs}
\end{figure}

% python
No todos los servicios del trabajo se realizaron con tecnologías relacionadas a JavaScript.
Particularmente se utilizaron una serie de bibliotecas y herramientas basadas en Python, que es un lenguaje de programación del tipo interpretado.
Este lenguaje posee una gran cantidad de recursos hechos por terceros y que se encuentran a disposición con licencias libres.
Si bien el lenguaje no fue visto con profundidad en la cursada, fue suficiente para entender su potencial, ya que muchas de sus bibliotecas fueron útiles para desarrollar algunos de los servicios más pequeños del trabajo, que no justificaron el uso de una plataforma como Nodejs.

% mosquitto
Para implementar el protocolo MQTT se decidió utilizar paquetes y bibliotecas desarrollados por terceros.
La primera herramienta fue Eclipse Mosquitto que es un broker del protocolo.
Es liviano, no demanda grandes recursos y puede ser ejecutado en ordenadores monoplaca.
Además es flexible ya que permite ser configurado de distintas maneras según las necesidades de la aplicación.

Existen varios niveles de calidad de servicio para seleccionar y también se pueden agregar usuarios de manera opcional.
Los usuarios poseen distintos permisos para publicar y suscribirse a diferentes \emph{topics}.
También es posible utilizar una capa de seguridad en los mensajes que se basa en la encriptación con llaves privadas y el uso de llaves públicas para realizar las lecturas.
Uno de los atractivos de la plataforma es la serie de programas utilitarios que incluye como recursos adicionales y que se integran en la terminal del sistema operativo.
Estas aplicaciones permiten realizar publicaciones y subscripciones para correr pruebas y además se pueden crear contraseñas encriptadas para los usuarios.

\subsection{Bases de datos}
% mongodb
Para persistir la información durante la ejecución del trabajo se utilizó MongoDB.
"MongoDB es una base de datos distribuida, basada en documentos y de uso general diseñada para desarrolladores de aplicaciones modernas y para la era de la nube" \citep{WEBSITE:MongoHome}.
Se decidió utilizar esta tecnología para la capa de procesamiento debido a la afinidad que posee para realizar aplicaciones IoT.
Además es la base de datos documental más utilizada actualmente \citep{WEBSITE:DBRanking}.
Por lo tanto se evaluó como aspecto relevante ganar experiencia con esta tecnología.

% redis
Los contenedores que forman la aplicación deben compartir una memoria volátil para que el sistema pueda funcionar correctamente.
Se eligió a Redis para que funcione como memoria compartida entre los componentes aislados dentro del trabajo.
Redis \citep{da2015redis} es un almacén de estructuras de datos en memoria y puede ser usado como una base de datos, una \emph{caché} de memoria y un broker para crear un bus de eventos.
Esta tecnología permite hacer persistencia de datos pero solo en el esquema de clave-valor y permite también establecer los tiempos de vida para los datos.
Los tiempos de vida de los datos se utilizaron en el trabajo para gestionar las sesiones de los usuarios.
Si una persona pasa demasiado tiempo sin interactuar con la aplicación tiene que volver a ingresar su usuario y contraseña para seguir utilizando el sistema con normalidad.	
	
% Descripción de las dependencias del trabajo
\section{Bibliotecas y paquetes de terceros}

\subsection{Bibliotecas y paquetes web}
% Angular & Angular Material
Para implementar la capa de aplicación se utilizó un framework de diseño de interfaces gráficas llamado Angular y se creó una SWA \citep{mesbah2007migrating}.
Se desarrolló el código usando el lenguaje TypeScript que es un \emph{superset} de JavaScript.
Está basado en el paradigma de componentes para generar aplicaciones escalables y poder reutilizar el código escrito en otros trabajos.
Además pone a disposición una colección de bibliotecas para manejar formularios, comunicaciones cliente-servidor, \emph{routing}, entre otras funcionalidades.
Tiene entre sus facilidades ofrecidas, un cliente de terminal que acelera los tiempos de desarrollo.
Por todas estas ventajas, se decidió inclinarse por este framework.

El estilo empleado en el trabajo fue el de diseño material.
Para lograrlo se trabajó con Angular Material \citep{WEBSITE:AngularMaterialio} que es una biblioteca de componentes de interfaz gráfica para Angular.
Está pensada para construir una experiencia consistente y funcional, adhiriendo con los principios de diseño más utilizados.
Estos principios se refieren a la portabilidad entre navegadores y la independencia de dispositivos.

% express & cors
Se necesitó crear una \emph{REST API} que es una interfaz para lograr una transferencia de datos a través de una representación de estados \citep{zhou2014rest}.
Para lograrlo se usó un framework para realizar servidores con Nodejs denominado Express \citep{WEBSITE:ExpressPage}.
El framework está pensado para construir aplicaciones webs e interfaces de programación de aplicaciones (API).
Una API se encarga de definir las interacciones entre múltiples programas o puede intermediar entre componentes de hardware y software.
Express es la biblioteca más utilizada para trabajar con Nodejs. \citep{WEBSITE:Express}

Para solucionar el problema de intercambio de recursos de origen cruzado del frontend (CORS) \citep{WEBSITE:CORS} se utilizó la biblioteca CORS para Nodejs.
CORS \citep{WEBSITE:CORSjs} es una solución que permite acceder a recursos restringidos de un dominio diferente al del frontend.

% wsjs
Para cumplir con el requerimiento de la sección \ref{objetivos} que demanda ver en vivo las mediciones al calibrar un instrumento, se utilizó WS \citep{WEBSITE:WSjs}. 
Esta dependencia es una biblioteca de Nodejs que se usa para realizar servidores con la capacidad de entablar una conexión WebSocket.
Es fácil de utilizar y es altamente configurable.
Se decidió incorporarla al trabajo debido a que la documentación es completa y simple de entender.

\subsection{Bibliotecas y paquetes de transporte}
% paho MQTT
Eclipse Paho \citep{WEBSITE:EclipsePaho} es una biblioteca MQTT que está disponible para varios lenguajes de programación.
En el trabajo realizado se utilizó para darle conectividad a los servicios programados en Python.
La biblioteca provee una clase cliente que habilita la comunicación con un broker MQTT.
Se ofrece además, una serie de funciones auxiliares que permiten codificar el programa con mayor facilidad.

% mqttjs
Se necesitó un recurso que ofrezca de capacidades MQTT al código escrito en JavaScript y para esa misión se usó MQTTjs \citep{WEBSITE:MQTTjs}.
MQTTjs es una biblioteca que provee de las herramientas para crear un cliente MQTT en Nodejs y en un navegador.
Se lo puede instalar de forma global en el sistema operativo para hacer uso de las herramientas de terminal que ofrece.
Las herramientas permiten hacer pruebas de subscripción y publicación de mensajes.

% pyModbusTCP
Para que el código escrito en Python pueda utilizar el protocolo ModbusTCP se usó la biblioteca PyModbusTCP \citep{WEBSITE:pyModbusTCP}.
Este recurso permite crear un cliente para acceder a un servidor o bien crear una aplicación que se comporte como esclavo.

% oitc/modbus-server
Oitc/modbus-server es un servidor ModbusTCP realizado en Python y disponible como imagen de Docker.
Está configurado para utilizar el puerto 5020 para evitar problemas de permisos con el sistema operativo \citep{WEBSITE:dockerhubModbus}.
El sistema de orquestación corrige el puerto al conectar el 5020 del contenedor con el 502 del ordenador.

% mongoose
Mongoose \citep{WEBSITE:mongoose} es una biblioteca de modelado de datos de objetos(ODM).
Un ODM gestiona las relaciones entre los datos, provee de una validación de esquema y se usa para traducir los objetos del programa en ejecución en una representación dentro de la base de datos.
La biblioteca se creó para trabajar con MongoDB y está disponible para la plataforma Nodejs.

\subsection{Bibliotecas de seguridad y pruebas}
% bcrypt jsonwebtoken
Se necesitó cumplir con las necesidades de seguridad que se enumeraron en la sección \ref{objetivos}.
Para lograrlo se usaron dos bibliotecas compatibles con Nodejs.
Bcrypt es una biblioteca que contiene funciones para encriptar contraseñas.
La encriptación se basa en un esquema de \emph{hashing} e incorpora una \emph{salt} para proteger los datos de un ataque \emph{rainbow table}.
Para darle resistencia a los ataques de búsqueda por fuerza bruta, la biblioteca implementa una función adaptativa que por cada iteración se vuelve más lenta.
Esto hace que su resistencia sea fuerte aún cuando el ordenador que realiza el ataque sea potente.
Por estas razones se decidió usar este recurso en el trabajo, con la idea de proteger las contraseñas de los usuarios evitando que persistan como texto plano.

% jsonwebtoken
La segunda biblioteca utilizada para cumplir con los requerimientos de seguridad fue Jsonwebtoken.
Es un paquete que permite crear un \emph{JavaScript Web Token (JWT)}.
JWT es un estándar que se utiliza para la fabricación de \emph{tokens} de acceso que permiten identificar una entidad y determinar cuales son sus privilegios en el sistema \citep{olivera2014jwt}.
El token está formado por una cabecera, una carga útil y una firma.
La cabecera identifica el algoritmo de encriptación y el tipo de token.
La carga útil lleva consigo la información relevante para el funcionamiento de la aplicación.
Finalmente la firma cierra el token para certificar la llave privada del servidor.
Es relevante mencionar que la biblioteca está diseñada para funcionar con Nodejs

% chai & mocha
La biblioteca Chai ofrece herramientas para hacer pruebas del software escrito.
Fue diseñada para Nodejs y puede ser integrada a cualquier framework de JavaScript.
En el trabajo fue combinado con Mocha, que es un framework de automatización de pruebas para Nodejs.
Las pruebas obtenidas al combinar estos dos recursos fueron fundamentales para detectar comportamientos no deseados.

Todas los recursos externos del trabajo se pueden visualizar en la tabla \ref{tab:dependencias}.

\begin{table}[h]
	\centering
	\caption{\label{tab:dependencias}Dependencias del trabajo.}
	\begin{tabular}{c c}
		\toprule
		\textbf{Recurso externo}      & \textbf{Función}                          \\
		\midrule
		Docker             & Motor de contenedores                     \\
		Docker Compose     & Orquestación de contenedores              \\
		Dockerfile         & Creación de imágenes para Docker          \\
		Nodejs             & Servidor para JavaScript                  \\
		Python             & Lenguaje para los servicios ligeros       \\
		Eclipse Mosquitto  & Broker MQTT de la aplicación              \\
		MongoDB            & Base de datos                             \\
		Redis              & Memoria compartida entre contenedores     \\
		Angular            & Framework para crear la SWA               \\
		Angular Material   & Componentes gráficos para Angular         \\
		Express            & REST API para Nodejs                      \\
		CORS               & Intercambio de recursos de origen cruzado \\
		Paho MQTT          & Biblioteca MQTT para Python               \\
		MQTTjs             & Biblioteca MQTT para Nodejs               \\
		WS                 & Funcionalidad WebSocket para Nodejs       \\
		Mongoose           & ODM para Nodejs y MongoDB                 \\
		Bycrypt            & Seguridad de contraseñas                  \\
		JsonWebToken       & Seguridad de sesiones de usuarios         \\
		Chai               & Pruebas unitarias para JavaScript         \\
		Mocha              & Pruebas automáticas para Nodejs           \\
		PyModbusTCP        & Biblioteca Modbus para Python             \\
		Oitc/modbus-server & Servidor Modbus                           \\
		\bottomrule
		\hline
	\end{tabular}
\end{table}