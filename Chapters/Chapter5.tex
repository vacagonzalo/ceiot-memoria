% Chapter Template

\chapter{Conclusiones} % Main chapter title

\label{Chapter5} % Change X to a consecutive number; for referencing this chapter elsewhere, use \ref{ChapterX}


%----------------------------------------------------------------------------------------

%----------------------------------------------------------------------------------------
%	SECTION 1
%----------------------------------------------------------------------------------------

Este capítulo trata sobre el valor agregado que se le dio al cliente, el aprendizaje adquirido y los siguientes pasos a seguir.

% Valor agregado al cliente y mi aprendizaje
\section{Resultados obtenidos}

El trabajo logró cumplir con los requerimientos que solicitó el cliente, y fueron listados en el capítulo \ref{Chapter1}.
Esta situación sirvió para entablar una relación positiva con el departamento de ingeniería de Gador, ya que el cliente manifestó su conformidad con los resultados obtenidos.

La planificación original del trabajo se pudo cumplir pero solo incrementado la cantidad de horas dedicadas.
El principal motivo de retraso que demandó una mayor dedicación horaria fue la poca información sobre el sistema propietario en planta.
Además, durante la cursada de la especialización se vieron temas de testeo de software que hicieron visibles ciertas fallas del trabajo.
Se dedicó un gran esfuerzo para depurar el código y lograr así una calidad de producción.

La imposibilidad de realizar pruebas dentro de la infraestructura de Gador fue un riesgo que lamentablemente se manifestó.
Solo pudo ser sorteado utilizando una máquina virtual con una licencia de uso único de seis horas de duración para verificar la comunicación.
El riesgo que por fortuna no se hizo realidad fue que alguna de las personas necesarias para realizar el trabajo se enfermara de \emph{covid-19}, o que se tomaran decisiones de prevención que afectaran la normalidad del desarrollo.

Las técnicas que mejor resultado dieron durante la creación del trabajo fue la automatización y despliegue de contenedores usando \emph{Docker} y \emph{Docker Compose} y el desarrollo orientado a pruebas.
La combinación de estos conocimientos genera software de calidad de producción que puede ser desplegado con gran facilidad en múltiples plataformas.

% Siguientes pasos para implementar el trabajo dentro del ambiente productivo de Gador
\section{Trabajo futuro}

La principal mejora a realizar es en la capa de negocios, que si bien cumple con los requerimientos del cliente, tendría un salto de valor incorporar \emph{Checkmk} al sistema.
Finalmente quedaría incorporar el trabajo a la infraestructura de Gador, para lograrlo se debe crear el hardware necesario.
El siguiente paso natural es iniciar un proyecto para diseñar los sensores y puntos de agregación para tener una solución completa.